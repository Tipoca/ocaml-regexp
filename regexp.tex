\documentclass[a4paper,twoside]{report}

\usepackage[T1]{fontenc}
\usepackage[latin1]{inputenc}
\usepackage{times}
\usepackage{url}

%BEGIN LATEX
\setlength{\textwidth}{16cm}
\setlength{\textheight}{23cm}
\setlength{\oddsidemargin}{0cm}
\setlength{\evensidemargin}{0cm}
\setlength{\topmargin}{0cm}
\renewcommand{\textfraction}{0.01}
\renewcommand{\topfraction}{.99}
%END LATEX

\usepackage[noweb]{ocamlweb}
\usepackage{graphics}

\renewcommand{\ocwbt}[1]{\textsl{#1}}
\renewcommand{\ocwlowerid}[1]{\textsl{#1\/}}
\renewcommand{\ocwupperid}[1]{\textsl{#1\/}}
\renewcommand{\ocwmodule}[1]{\section{Implementation of module \ocwupperid{#1}}}
\renewcommand{\ocwinterface}[1]{\section{Module \ocwupperid{#1}}}

%BEGIN LATEX
\renewcommand{\ocwbeginindex}{\renewcommand{\indexname}{Index of Identifiers}\begin{theindex}}
%END LATEX
%HEVEA\renewcommand{\ocwbeginindex}{\section{Index of Identifiers}\addcontentsline{toc}{section}{\indexname}\begin{itemize}}
%HEVEA\renewcommand{\ocwendindex}{\end{itemize}}

\begin{document}
\sloppy

\title{A simple library for regular expressions}
\author{Claude March�}
\date{\today}

\maketitle

\tableofcontents

\chapter{Introduction}

This library implements simple use of regular expressions. It provides
direct definitions of regular expressions form the usual
constructions, or definition of such expressions form a syntactic
manner using GNU regexp syntax. It provides a simple compilation of
regexp into a deterministic automaton, and use of such an automaton for
matching and searching.

\chapter{Library documentation}

This is the documentation for use of the library. It provides all
functions in one module.

\input{library}

\chapter{Documentation of implementation}

This part describe the implementation of the library. It provides
several modules which depends each other as shown by the graph below.
 
\begin{center}
\includegraphics{dependency-graph.eps}
\end{center}

\input{implem}

\end{document}

%%% Local Variables: 
%%% mode: latex
%%% TeX-master: t
%%% End: 
